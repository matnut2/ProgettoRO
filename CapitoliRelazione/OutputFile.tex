All'interno della cartella \textbf{AMPL} si può trovare un file denominato \textit{Ospedale.FileSave.run} che esegue in serie tutti e tre gli scenari sopracitati e ne salva l'output in un file dedicato chiamato \textit{Output.txt}.
Per fare ciò, sono sono stati utilizzati gli operatori $>$ e $>>$. Il primo serve a creare un nuovo file (o a ripulirlo se già esistente) e scriverci l'istruzione della stessa linea; il secondo serve invece per aprire il file e fare un \textit{append} dell'istruzione di riferimento.