\subsection{Insiemi}
\begin{itemize}
    \item $Giorni$: giorni della settimana in cui è attivo il servizio
    \item $Fornitori$: fornitori che possono fornire ambulanze
    \item $Tipo$: tipi di ambulanze disponibili (indica la destinazione d'uso del mezzo nel contesto)
\end{itemize}
\subsection{Parametri}
\begin{itemize}
    \item $bisogno_{g, t}$: numero di ambulanze di tipo $t \in Tipo$ necessarie per il giorno $g \in Giorni$
    \item $surplus_{g, t}$: numero di ambulanze di scorta di tipo $t \in Tipo$ necessarie per il giorno $g \in Giorni$
    \item $maxAmbulanze_{t, f}$: numero massimo di ambulanze di tipo $t \in Tipo$ che il fornitore $f \in Fornitori$ può fornire al giorno
    \item $maxGiorni_{t, f}$: numero massimo di giorni in cui un fornitore $f \in Fornitori$ può fornire ambulanze di tipo $t \in Tipo$
    \item $costoGiornaliero_{t, f}$: costo giornaliero per l'utilizzo di un'ambulanza di tipo $t \in Tipo$ del fornitore $f \in Fornitori$
    \item $addSurplus_{f}$: aumento percentuale del costo per l'utilizzo di un'ambulanza del fornitore $f \in Fornitori$ come surplus
    \item $costoAttivazioneSettimanale_{f}$: costo di attivazione del fornitore $f \in Fornitori$ per la settimana
    \item $BigM$: costante sufficientemente grande utilizzata per i vincoli logici
    \item $MIN\_FORNITORI$: constante che indica il numero minimo di fornitori da attivare al giorno
\end{itemize}
\subsection{Variabili}
\begin{itemize}
    \item $ambulanze_{t, f, g}$: numero di ambulanze di tipo $t \in Tipo$ del fornitore $f \in Fornitori$ attivate il giorno $g \in Giorni$
    \item $ambulanzeSurplus_{t, f, g}$: numero di ambulanze di tipo $t \in Tipo$ del fornitore $f \in Fornitori$ attivate il giorno $g \in Giorni$ come surplus
    \item $attivazioneSettimanale_{f}$ = 
    \(
    \begin{cases}
        1 & \text{se viene attivato il fornitore f per la settimana} \\
        0 & altrimenti
    \end{cases}\)
    \item $attivazioneGiornaliera_{t, f, g}$ = 
    \(
    \begin{cases}
        1 & \text{se viene attivata l'ambulanza di tipo t del fornitore f il giorno g} \\
        0 & altrimenti
    \end{cases}\)
\end{itemize}
\subsection{Funzione Obiettivo}
La funzione obiettivo del problema può essere espressa in forma generale come segue:
\begin{align*}
	\textrm{min} \quad & \textbf{Costo\ Giornaliero\ Ambulanze\ Tipo\ A} + \textbf{Costo\ Giornaliero\ Ambulanze\ Tipo\ B} \\ 
    & + \textbf{Costo\ Attivazione\ Ambulanze\ Tipo\ A} + \textbf{Costo\ Attivazione\ Ambulanze\ Tipo\ B}
\end{align*}
Dove i termini sono così definiti:
\begin{flalign*}
    & \textbf{Costo\ Giornaliero\ Ambulanze\ Tipo\ A} = \sum_{f \in FornitoriA, g \in Giorni} ambulanzeA[f, g] \cdot costoGiornalieroA[f] && \\
    & \textbf{Costo\ Giornaliero\ Ambulanze\ Tipo\ B} = \sum_{f \in FornitoriB, g \in Giorni} ambulanzeB[f, g] \cdot costoGiornalieroB[f] && \\
    & \textbf{Costo\ Attivazione\ Ambulanze\ Tipo\ A} = \sum_{f \in FornitoriA} attivazioneSettimanaleA[f] \cdot costoAttivazioneA[f] && \\
    & \textbf{Costo\ Attivazione\ Ambulanze\ Tipo\ B} = \sum_{f \in FornitoriB} attivazioneSettimanaleB[f] \cdot costoAttivazioneB[f] &&
\end{flalign*}
\textbf{subject to}
\begin{itemize}
    \item Ogni giorno la Centrale Operativa ha necessità di un numero diverso di ambulanze:
\end{itemize}
\begin{flalign*}
    &  \sum_{f \in FornitoriA} ambulanzeA[f, g] \geq bisognoA[g] + surplusA[g] & \forall g \in Giorni & \\
    &  \sum_{f \in FornitoriB} ambulanzeB[f, g] \geq bisognoB[g] + surplusB[g] & \forall g \in Giorni &
\end{flalign*}
\begin{itemize}
    \item I fornitori dispongono di un numero massimo di ambulanze fornibili durante la settimana:
\end{itemize}
\begin{flalign*}
    &  ambulanze[f, g] \leq maxA[f] & \forall f \in FornitoriA, g \in Giorni & \\
    &  ambulanze[f, g] \leq maxB[f] & \forall f \in FornitoriB, g \in Giorni &
\end{flalign*}
\begin{itemize}
    \item Le ambulanze di un determinato fornitore si possono attivare se e solo se si è pagata la quota di attivazione settimanale:
\end{itemize}
\begin{flalign*}
    & ambulanzeA[f, g] \leq BigM \cdot attivazioneSettimanaleA[f] & \\
    & ambulanzeB[f, g] \leq BigM \cdot attivazioneSettimanaleB[f] &
\end{flalign*}
\textbf{Domini:}
\begin{itemize}
    \item $ambulanzeA_{f, g}, \ ambulanzeB_{f, g} \ \geq \ 0$
    \item $attivazioneSettimanaleA_{f}, \ attivazioneSettimanaleB_{f} \ \in \ {0, \ 1}$
\end{itemize}