\subsection{Insiemi}
\begin{itemize}
    \item $Giorni$: giorni della settimana in cui è attivo il servizio
    \item $FornitoriA$: fornitori che dispongono di ambulanze di tipo A da attivare
    \item $FornitoriB$: fornitori che dispongono di ambulanze di tipo B da attivare
\end{itemize}
\subsection{Parametri}
\begin{itemize}
    \item $bisognoA_{g}$: fabbisogno per il giorno $g$ di ambulanze di tipo A (escluso surplus arbitrario)
    \item $bisognoB_{g}$: fabbisogno per il giorno $g$ di ambulanze di tipo B (escluso surplus arbitrario)
    \item $surplusA_{g}$: surplus di ambulanze di tipo A per il giorno $g$
    \item $surplusB_{g}$: surplus di ambulanze di tipo B per il giorno $g$
    \item $maxA_{fa}$: numero massimo di ambulanze di tipo A che il fornitore $fa$ può fornire
    \item $maxB_{fb}$: numero massimo di ambulanze di tipo B che il fornitore $fb$ può fornire
    \item $costoGiornalieroA_{fa}$: costo per l'attivazione giornaliera per una singola ambulanza di tipo A del fornitore $fa$
    \item $costoGiornalieroB_{fb}$: costo per l'attivazione giornaliera per una singola ambulanza di tipo B del fornitore $fb$
    \item $costoAttivazioneA_{fa}$: costo settimanale per l'attivazione di un fornitore $fa$ per la fornitura settimanale di ambulanze di tipo A
    \item $costoAttivazioneB_{fb}$: costo settimanale per l'attivazione di un fornitore $fb$ per la fornitura settimanale di ambulanze di tipo B
    \item $BigM$: utilizzato per vincolare l'utilizzo delle ambulanze rispetto all'attivazione del fornitore (costante sufficientemente grande), Questo, nei file \textbf{.dat} è stato definito utilizzando la più vicina potenza del 2 rispetto a $Max(bisognoA, bisognoB) + Max(surplusA, surplusB)$.
\end{itemize}
\subsection{Variabili Decisionali}
\begin{itemize}
    \item $ambulanzeA_{fa, ga}$ = numero di ambulanze del fornitore $fa$ attivate il giorno $ga$;
    \item $ambulanzeB_{fb, gb}$ =  numero di ambulanze del fornitore $fb$ attivate il giorno $gb$;
    \item $attivazioneSettimanaleA_{fa}$ = 
    \(
    \begin{cases}
        1 & \text{se viene attivato il fornitore di ambulanze di tipo A \textit{fa} per la settimana} \\
        0 & altrimenti
    \end{cases}\)
    \item $attivazioneSettimanaleB_{fb}$ = 
    \(
    \begin{cases}
        1 & \text{se viene attivato il fornitore di ambulanze di tipo B \textit{fb} per la settimana} \\
        0 & altrimenti
    \end{cases}\)
\end{itemize}
\subsection{Funzione Obiettivo}
La funzione obiettivo del problema può essere espressa in forma generale come segue:
\begin{align*}
	\textrm{min} \quad & \textbf{Costo\ Giornaliero\ Ambulanze\ Tipo\ A} + \textbf{Costo\ Giornaliero\ Ambulanze\ Tipo\ B} \\ 
    & + \textbf{Costo\ Attivazione\ Ambulanze\ Tipo\ A} + \textbf{Costo\ Attivazione\ Ambulanze\ Tipo\ B}
\end{align*}
Dove i termini sono così definiti:
\begin{flalign*}
    & \textbf{Costo\ Giornaliero\ Ambulanze\ Tipo\ A} = \sum_{f \in FornitoriA, g \in Giorni} ambulanzeA[f, g] \cdot costoGiornalieroA[f] && \\
    & \textbf{Costo\ Giornaliero\ Ambulanze\ Tipo\ B} = \sum_{f \in FornitoriB, g \in Giorni} ambulanzeB[f, g] \cdot costoGiornalieroB[f] && \\
    & \textbf{Costo\ Attivazione\ Ambulanze\ Tipo\ A} = \sum_{f \in FornitoriA} attivazioneSettimanaleA[f] \cdot costoAttivazioneA[f] && \\
    & \textbf{Costo\ Attivazione\ Ambulanze\ Tipo\ B} = \sum_{f \in FornitoriB} attivazioneSettimanaleB[f] \cdot costoAttivazioneB[f] &&
\end{flalign*}
\textbf{subject to}
\begin{itemize}
    \item Ogni giorno la Centrale Operativa ha necessità di un numero diverso di ambulanze:
\end{itemize}
\begin{flalign*}
    &  \sum_{f \in FornitoriA} ambulanzeA[f, g] \geq bisognoA[g] + surplusA[g] & \forall g \in Giorni & \\
    &  \sum_{f \in FornitoriB} ambulanzeB[f, g] \geq bisognoB[g] + surplusB[g] & \forall g \in Giorni &
\end{flalign*}
\begin{itemize}
    \item I fornitori dispongono di un numero massimo di ambulanze fornibili durante la settimana:
\end{itemize}
\begin{flalign*}
    &  ambulanze[f, g] \leq maxA[f] & \forall f \in FornitoriA, g \in Giorni & \\
    &  ambulanze[f, g] \leq maxB[f] & \forall f \in FornitoriB, g \in Giorni &
\end{flalign*}
\begin{itemize}
    \item Le ambulanze di un determinato fornitore si possono attivare se e solo se si è pagata la quota di attivazione settimanale:
\end{itemize}
\begin{flalign*}
    & ambulanzeA[f, g] \leq BigM \cdot attivazioneSettimanaleA[f] & \\
    & ambulanzeB[f, g] \leq BigM \cdot attivazioneSettimanaleB[f] &
\end{flalign*}
\textbf{Domini:}
\begin{itemize}
    \item $ambulanzeA_{f, g}, \ ambulanzeB_{f, g} \ \geq \ 0$
    \item $attivazioneSettimanaleA_{f}, \ attivazioneSettimanaleB_{f} \ \in \ {0, \ 1}$
\end{itemize}