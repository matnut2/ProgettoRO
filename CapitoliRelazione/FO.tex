\subsection{Funzione Obiettivo}
La funzione obiettivo del problema può essere espressa in forma generale come segue:
\begin{align*}
	\textrm{min} \quad & \textbf{Costo\ Giornaliero\ Ambulanze\ Standard} + \textbf{Costo\ Giornaliero\ Ambulanze\ Surplus} \\ 
    & + \textbf{Costo\ Attivazione\ Settimanale} + \textbf{Costo\ Attivazione\ Giornaliero}
\end{align*}
Dove i termini sono così definiti:
\begin{flalign*}
    & \textbf{Costo\ Giornaliero\ Ambulanze\ Standard} = \sum_{t \in Tipo, f \in Fornitori, g \in Giorni} ambulanze[t, f, g] \cdot costoGiornaliero[t, f] && \\
    & \textbf{Costo\ Giornaliero\ Ambulanze\ Surplus} = Costo\ Giornaliero\ Ambulanze\ Standard\ + && \\ 
    &         Costo Giornaliero Ambulanze Standard \cdot addSurplus[t, f] && \\
    & \textbf{Costo\ Attivazione\ Settimanale} = \sum_{f \in Fornitori} attivazioneSettimanale[f] \cdot costoAttivazioneSettimanale[f] && \\
    & \textbf{Costo\ Attivazione\ Giornaliera} = \sum_{t \in Tipo, f \in Fornitori, g \in Giorni} attivazioneGiornaliera[t, f, g] \cdot costoGiornaliero[t, f] &&
\end{flalign*}
\textbf{subject to}
\begin{itemize}
    \item Ogni giorno la Centrale Operativa ha necessità di un numero diverso di ambulanze:
\end{itemize}
\begin{flalign*}
    &  \sum_{t \in Tipo, f \in FornitoriA} ambulanze[t, f, g] \geq bisogno[t, g] & \forall g \in Giorni & \\
    &  \sum_{t \in Tipo, f \in FornitoriA} ambulanzeSurplus[t, f, g] \geq surplus[t, g] & \forall g \in Giorni & \\
\end{flalign*}
\begin{itemize}
    \item I fornitori dispongono di un numero massimo di ambulanze fornibili in un giorno:
\end{itemize}
\begin{flalign*}
    &  ambulanze[t, f, g] + ambulanzeSurplus[t, f, g] \leq maxAmbulanze[t, f] & \forall t \in Tipo, f \in Fornitori, g \in Giorni & \\
\end{flalign*}
\begin{itemize}
    \item Le ambulanze di un determinato fornitore si possono attivare se e solo se si è pagata la quota di attivazione settimanale:
\end{itemize}
\begin{flalign*}
    & ambulanze[t, f, g] + ambulanzeSurplus[t, f, g] \leq BigM \cdot attivazioneSettimanale[f] && \\
\end{flalign*}
\begin{itemize}
    \item Ogni giorno bisogna attivare un numero minimo di fornitori:
\end{itemize}
\begin{flalign*}
    & \sum_{f \in Fornitori} attivazioneGiornaliera[t, f, g] >= MIN\_FORNITORI\ \ \  \forall t \in Tipo, g \in Giorni && \\
    & attivazioneGiornaliera[t, f, g] <= attivazioneSettimanale[f] \ \ \ \  \forall t \in Tipo, f \in Fornitori, g \in Giorni && \\
    & ambulanze[t, f, g] + ambulanzeSurplus[t, f, g] <= BigM \cdot attivazioneGiornaliera[t, f, g] && \\
    &  \ \ \ \forall t \in Tipo, f \in Fornitori, g \in Giorni && \\
\end{flalign*}
\begin{itemize}
    \item Ogni fornitore può essere attivato un numero massimo di volte durante la settimana
\end{itemize}
\begin{flalign*}
    & \sum_{g \in Giorni} attivazioneGiornaliera[t, f, g] <= maxGiorni[t, f] & \forall t \in Tipo, f \in Fornitori & \\
\end{flalign*}
\textbf{Domini:}
\begin{itemize}
    \item $ambulanze_{t, f, g} \geq \ 0$
    \item $ambulanzeSurplus_{t, f, g} \geq 0$
    \item $attivazioneSettimanale_{f} \in \ {[0, \ 1]}$
    \item $attivazioneGiornaliera_{t, f, g} \in \ {[0, \ 1]}$
\end{itemize}