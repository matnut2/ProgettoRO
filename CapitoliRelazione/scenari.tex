\textbf{Premessa: } dato che negli scenari proposti gli insiemi non variano, questi saranno di seguito presentati una sola volta.
Gli insiemi sono così definiti: \newline \vspace*{0.1 cm} \newline
\begin{tabular}{|| c || c c c c c c c ||}
    \hline
    Giorni & lunedi & martedi & mercoledi & giovedi & venerdi & sabato & domenica \\ 
    \hline
\end{tabular}
\newline
\vspace*{0.3 cm}
\newline
\begin{tabular}{|| c || c c c ||}
    \hline
        FornitoriA & rossa & verde & interna \\
    \hline
    
\end{tabular}
\newline
\vspace*{0.3 cm}
\newline
\begin{tabular}{|| c || c c c ||}
    \hline
        FornitoriA & bianca & verde & interna \\
    \hline
    
\end{tabular}

\subsection{Primo Scenario}
\subsubsection{Descrizione}
In questo primo scenario, i dati sono stati inseriti manualmente, con un numero di ambulanze necessarie tale da permettere di trovare una soluzione ottima.
\subsubsection{Dati}
\begin{tabular}{| c | c |}
    \hline
    \multicolumn{2}{|c|}{BisognoA}\\
    \hline \hline
    lunedi & 20 \\
    martedi & 22 \\
    mercoledi & 10 \\
    giovedi & 20 \\
    venerdi & 30 \\
    sabato & 30 \\
    domenica & 15 \\
    \hline
\end{tabular}
\begin{tabular}{| c | c |}
    \hline
    \multicolumn{2}{|c|}{BisognoB}\\
    \hline \hline
    lunedi & 3 \\
    martedi & 5 \\
    mercoledi & 2 \\
    giovedi & 3 \\
    venerdi & 4 \\
    sabato & 2 \\
    domenica & 0 \\
    \hline
\end{tabular}
\begin{tabular}{| c | c |}
    \hline
    \multicolumn{2}{|c|}{surplusA}\\
    \hline \hline
    lunedi & 3 \\
    martedi & 3 \\
    mercoledi & 3 \\
    giovedi & 2 \\
    venerdi & 5 \\
    sabato & 6 \\
    domenica & 4 \\
    \hline
\end{tabular}
\begin{tabular}{| c | c |}
    \hline
    \multicolumn{2}{|c|}{surplusB}\\
    \hline \hline
    lunedi & 1 \\
    martedi & 3 \\
    mercoledi & 1 \\
    giovedi & 1 \\
    venerdi & 2 \\
    sabato & 1 \\
    domenica & 1 \\
    \hline
\end{tabular}
\begin{tabular}{| c | c |}
    \hline
    \multicolumn{2}{|c|}{maxA}\\
    \hline \hline
    rossa & 15 \\
    verde & 30 \\
    interna & 3 \\
    \hline    
\end{tabular}
\begin{tabular}{| c | c |}
    \hline
    \multicolumn{2}{|c|}{maxB} \\
    \hline \hline
    bianca & 5 \\
    verde & 10 \\
    interna & 8 \\
    \hline    
\end{tabular}
\begin{tabular}{| c | c |}
    \hline
    \multicolumn{2}{|c|}{costoGiornalieroA}\\
    \hline \hline
    rossa & 15 \\
    verde & 18 \\
    interna & 8 \\
    \hline    
\end{tabular}
\begin{tabular}{| c | c |}
    \hline
    \multicolumn{2}{|c|}{costoGiornalieroB} \\
    \hline \hline
    bianca & 20 \\
    verde & 12 \\
    interna & 5 \\
    \hline    
\end{tabular}
\begin{tabular}{| c | c |}
    \hline
    \multicolumn{2}{|c|}{costoAttivazioneA}\\
    \hline \hline
    rossa & 130 \\
    verde & 50 \\
    interna & 0 \\
    \hline    
\end{tabular}
\begin{tabular}{| c | c |}
    \hline
    \multicolumn{2}{|c|}{costoAttivazioneB} \\
    \hline \hline
    bianca & 70 \\
    verde & 55 \\
    interna & 0 \\
    \hline    
\end{tabular}


\subsubsection{Output}
L'esecuzione del file \textbf{.run} mostra che il costo minimo è di \texteuro$2929$.

\subsection{Secondo Scenario}
\subsubsection{Descrizione}
Il secondo set di dati differisce dal primo solo per la richiesta di ambulanze sia di tipo A che di tipo B per il lunedi che non è soddisfacibile. Per questo motivo, nel prossimo paragrafo saranno riportati solo i parametri modificati.
\subsubsection{Dati}
\begin{tabular}{| c | c |}
    \hline
    \multicolumn{2}{|c|}{BisognoA}\\
    \hline \hline
    lunedi & 150 \\
    martedi & 22 \\
    mercoledi & 10 \\
    giovedi & 20 \\
    venerdi & 30 \\
    sabato & 30 \\
    domenica & 15 \\
    \hline
\end{tabular}
\begin{tabular}{| c | c |}
    \hline
    \multicolumn{2}{|c|}{BisognoB}\\
    \hline \hline
    lunedi & 65 \\
    martedi & 5 \\
    mercoledi & 2 \\
    giovedi & 3 \\
    venerdi & 4 \\
    sabato & 2 \\
    domenica & 0 \\
    \hline
\end{tabular}
\begin{tabular}{| c | c |}
    \hline
    \multicolumn{2}{|c|}{maxA}\\
    \hline \hline
    rossa & 15 \\
    verde & 30 \\
    interna & 3 \\
    \hline    
\end{tabular}
\begin{tabular}{| c | c |}
    \hline
    \multicolumn{2}{|c|}{maxB} \\
    \hline \hline
    bianca & 5 \\
    verde & 10 \\
    interna & 8 \\
    \hline    
\end{tabular}
\subsubsection{Output}
Ovviamente, il costo minimo trovato eseguendo il file \textbf{.run} sarà pari a \texteuro$0$.

\subsection{Terzo Scenario}
\subsubsection{Descrizione}
Nel terzo scenario sono stati definiti soltanto gli insiemi, così da poter sfruttare i valori di default. Per questo motivo non verrà riportato il valore dei vari parametri in questo specificato nel modello.
\subsubsection{Output}
Eseguendo il file \textbf{.run} dedicato, utilizzando quindi i dati di default, il costo minimo risulta essere di \texteuro$350$.