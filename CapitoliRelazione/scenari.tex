\subsection{Primo Scenario}
\subsubsection{Descrizione}
In questo primo scenario, i dati sono stati inseriti manualmente, con un numero di ambulanze necessarie tale da permettere di trovare una soluzione ottima.
\subsubsection{Codice}
\begin{lstlisting}
### INSIEMI ###
set Giorni := lunedi martedi mercoledi giovedi venerdi sabato domenica;
set FornitoriA := rossa verde interna;
set FornitoriB := bianca verde interna;

### PARAMETRI ###
param bisognoA := 
lunedi 20 martedi 22 mercoledi 10 
giovedi 20 venerdi 30 sabato 30
domenica 15
;

param bisognoB := 
lunedi 3 martedi 5 mercoledi 2
giovedi 3 venerdi 4 sabato 2
domenica 0
;

param surplusA :=
lunedi 3 martedi 3 mercoledi 3
giovedi 2 venerdi 5 sabato 6
domenica 4
;

param surplusB :=
lunedi 1 martedi 3 mercoledi 1
giovedi 1 venerdi 2 sabato 1
domenica 1
;

param maxA :=
rossa 15 verde 20 interna 3
;

param maxB :=
bianca 5 verde 10 interna 8
;

param costoGiornalieroA :=
rossa 15 verde 18 interna 8
;

param costoGiornalieroB :=
bianca 20 verde 12 interna 5
;

param costoAttivazioneA :=
rossa 130 verde 50 interna 0
;

param costoAttivazioneB :=
bianca 70 verde 55 interna 0
; 

param BigM := 64;
\end{lstlisting}
\subsubsection{Output}
L'esecuzione del file \textbf{.run} mostra che il costo minimo è di \texteuro$2929$.

\subsection{Secondo Scenario}
\subsubsection{Descrizione}
Il secondo set di dati è stato costruito manualmente con una richiesta di ambulanze non soddisfacibile, impedendo quindi di trovare una soluzione.
\subsubsection{Codice}
\begin{lstlisting}
### INSIEMI ###
set Giorni := lunedi martedi mercoledi giovedi venerdi sabato domenica;
set FornitoriA := rossa verde interna;
set FornitoriB := bianca verde interna;

### PARAMETRI ###
param bisognoA := 
lunedi 150 martedi 22 mercoledi 10 
giovedi 20 venerdi 30 sabato 30
domenica 15
;

param bisognoB := 
lunedi 65 martedi 5 mercoledi 2
giovedi 3 venerdi 4 sabato 2
domenica 0
;

param surplusA :=
lunedi 3 martedi 3 mercoledi 3
giovedi 2 venerdi 5 sabato 6
domenica 4
;

param surplusB :=
lunedi 1 martedi 3 mercoledi 1
giovedi 1 venerdi 2 sabato 1
domenica 1
;

param maxA :=
rossa 15 verde 20 interna 3
;

param maxB :=
bianca 5 verde 10 interna 8
;

param costoGiornalieroA :=
rossa 15 verde 18 interna 8
;

param costoGiornalieroB :=
bianca 20 verde 12 interna 5
;

param costoAttivazioneA :=
rossa 130 verde 50 interna 0
;

param costoAttivazioneB :=
bianca 70 verde 55 interna 0
; 

param BigM := 256;
\end{lstlisting}
\subsubsection{Output}
Ovviamente, il costo minimo trovato eseguendo il file \textbf{.run} sarà pari a \texteuro$0$.

\subsection{Terzo Scenario}
\subsubsection{Descrizione}
Nel terzo scenario sono stati definiti soltanto gli insiemi, così da poter sfruttare i valori di default.
\subsubsection{Codice}
\begin{lstlisting}
### INSIEMI ###
set Giorni := lunedi martedi mercoledi giovedi venerdi sabato domenica;
set FornitoriA := rossa verde interna;
set FornitoriB := bianca verde interna;

### PARAMETRI ###
# Non definiti per test valori di default
\end{lstlisting}
\subsubsection{Output}
Eseguendo il file \textbf{.run} dedicato, utilizzando quindi i dati di default, il costo minimo risulta essere di \texteuro$350$.