\textbf{Premessa: } dato che negli scenari proposti gli insiemi non variano, questi saranno di seguito presentati una sola volta.
Gli insiemi sono così definiti: \newline \vspace*{0.1 cm} \newline
\begin{tabular}{|| c || c c c c c c c ||}
    \hline
    Giorni & lunedi & martedi & mercoledi & giovedi & venerdi & sabato & domenica \\ 
    \hline
\end{tabular}
\newline
\vspace*{0.3 cm}
\newline
\begin{tabular}{|| c || c c c c ||}
    \hline
        Fornitori & interna & bianca & rossa & verde \\
    \hline
\end{tabular}
\newline
\vspace*{0.3 cm}
\newline
\begin{tabular}{|| c || c c ||}
    \hline
        Tipo & trasporto & emergenza \\
    \hline
\end{tabular}

\subsection{Primo Scenario}
\subsubsection{Descrizione}
In questo primo scenario, i dati sono stati inseriti manualmente, con un numero di ambulanze necessarie tale da permettere di trovare una soluzione ottima e che sfrutti tutti i vincoli imposti.
\subsubsection{Dati}
\begin{center}
\begin{tabular}{|| c || c | c |}
    \hline
    \multicolumn{3}{|c|}{Bisogno}\\
    \hline \hline
    Giorno & Trasporto & Emergenza \\
    \hline
    lunedi & 3 & 20 \\
    martedi & 5 & 22 \\
    mercoledi & 2 & 10 \\
    giovedi & 3 & 20 \\
    venerdi & 4 & 30 \\
    sabato & 2 & 30 \\
    domenica & 0 & 15 \\
    \hline
\end{tabular}
\begin{tabular}{|| c || c | c |}
    \hline
    \multicolumn{3}{|c|}{Surplus}\\
    \hline \hline
    Giorno & Trasporto & Emergenza \\
    \hline
    lunedi & 1 & 3 \\
    martedi & 3 & 3 \\
    mercoledi & 1 & 3 \\
    giovedi & 1 & 2 \\
    venerdi & 2 & 5 \\
    sabato & 1 & 6 \\
    domenica & 1 & 4 \\
    \hline
\end{tabular}
\newline
\begin{tabular}{|| c || c | c |}
    \hline
    \multicolumn{3}{|c|}{maxAmbulanze}\\
    \hline \hline
    OdV & Trasporto & Emergenza \\
    \hline
    interna & 2 & 3 \\
    bianca & 5 & 0 \\
    rossa & 0 & 15 \\
    verde & 10 & 30 \\
    \hline    
\end{tabular}
\begin{tabular}{|| c || c | c |}
    \hline
    \multicolumn{3}{|c|}{maxGiorni}\\
    \hline \hline
    OdV & Trasporto & Emergenza \\
    \hline
    interna & 7 & 7 \\
    bianca & 3 & 0 \\
    rossa & 0 & 5 \\
    verde & 4 & 6 \\
    \hline    
\end{tabular}
\newline
\begin{tabular}{|| c || c | c |}
    \hline
    \multicolumn{3}{|c|}{costoGiornaliero}\\
    \hline \hline
    OdV & Trasporto & Emergenza \\
    \hline
    interna & 5 & 8 \\
    bianca & 10 & 0 \\
    rossa & 0 & 10 \\
    verde & 7 & 15 \\
    \hline    
\end{tabular}
\begin{tabular}{|| c || c |}
    \hline
    \multicolumn{2}{|c|}{addSurplus} \\ 
    \hline \hline
    OdV & Fattore Moltiplicativo \\
    \hline
    interna & 0.00 \\
    bianca  & 5.00 \\
    rossa   & 8.00 \\
    verde   & 10.00 \\
    \hline
\end{tabular}
\newline
\begin{tabular}{|| c || c |}
    \hline
    \multicolumn{2}{|c|}{costoAttivazioneSettimanale} \\ 
    \hline \hline
    OdV & Costo \\
    \hline
    interna & 25 \\
    bianca  & 150 \\
    rossa   & 130 \\
    verde   & 170 \\
    \hline
\end{tabular}
\end{center}
\subsubsection{Output}
L'esecuzione del file \textit{.run} relativo a questo scenario mostra che il costo minimo sostenibile è di \texteuro $3595$. Come si può vedere dall'output, ogni giorno sono stati attivati almeno 2 fornitori per tipo di ambulanza. Oltre a questo si può notare che, per quanto riguarda le ambulanze surplus, il programma ha scelto giustamente di usare quelle interne in quanto il fattore moltiplicativo per tale utilizzo è nullo.
\subsection{Secondo Scenario}
\subsubsection{Descrizione}
Il secondo set di dati prevede un bisogno e un surplus impossibili da soddisfare, per questo già il \textit{presolve} ritornerà un errore che indica l'impossibilità di trovare una soluzione con i vincoli di riferimento.
Senza riportare nuovamente le tabelle, il set di dati utilizza tutti i dati di default tranne che per i parametri \textit{bisogno} e \textit{surplus} che hanno un valore di 200.

\subsubsection{Output}
Ovviamente, il costo minimo trovato eseguendo il file \textbf{.run} sarà pari a \texteuro$0$. La non risolvibilità del problema è data dal fatto che l'insieme delle ambulanze dedicate alle emergenze ($20$) e dedicate al trasporto ($20$) non sono abbastanza per coprire il fabbisogno dell'intera settimana (ossia di $200$ ambulanze per tipo al giorno).

\subsection{Terzo Scenario}
\subsubsection{Descrizione}
Nel terzo scenario sono stati definiti soltanto gli insiemi, così da poter sfruttare i valori di default. Per questo motivo non verrà riportato il valore dei vari parametri in quanto specificato nel modello.
\subsubsection{Output}
Eseguendo il file \textbf{.run} dedicato, utilizzando quindi i dati di default, il costo minimo risulta essere di \texteuro$2720$.