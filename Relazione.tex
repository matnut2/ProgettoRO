\documentclass[10pt, a4paper]{article}
\usepackage[utf8]{inputenc}
\usepackage{hyperref}   % PER I LINK INTERNI ED ESTERNI
\usepackage{graphicx}   % PER LE IMMAGINI
\usepackage{amsmath}    % PER LE FRAZIONI
\usepackage{listings}   % PER IL CODICE
\usepackage{xcolor}     % PER COLORARE IL CODICE
\usepackage[margin=2.5cm]{geometry}

\graphicspath{{img/}}

\hypersetup{
    colorlinks=true,
    linkcolor=black,
    urlcolor=cyan,
    }

    \usepackage{courier}
    \usepackage{listings}
    
    \lstdefinelanguage{AMPL}{keywords={set,param,var,arc,integer,minimize,maximize,subject,to,node,sum,in,Current,complements,integer,solve_result_num,IN,contains,less,suffix,INOUT,default,logical,sum,Infinity,dimen,max,symbolic
    ,Initial,div,min,table,LOCAL,else,option,then,OUT,environ,setof ,union,all,exists,shell_exitcodeuntil,binary,forall,solve_exitcodewhile ,by,if,solve_messagewithin,check,in,solve_result
    },sensitive=true,comment=[l]{\#}}
    
    \lstset{frame=tb,
      language=AMPL,
      aboveskip=3mm,
      belowskip=3mm,
      showstringspaces=false,
      columns=flexible,
      basicstyle={\ttfamily},
      numbers=none,
      numberstyle=\tiny\color{gray},
      keywordstyle=\bfseries,
      commentstyle=\textit,
      stringstyle=\color{mauve},
      breaklines=true,
      breakatwhitespace=true,
      tabsize=3
    }
    

\title{Progetto di Ricerca Operativa}
\author{Soldà Matteo }
\date{A.A. 2022--2023}

\begin{document}

\begin{figure}
    \centering
    \includegraphics[width=0.50\textwidth]{Logo_Università_Padova}
\end{figure}

\maketitle

\newpage
\tableofcontents

\newpage
\section{Introduzione}
\subsection{Abstract}
La Centrale Operativa del 118 della città di Padova necessita di un gran numero di ambulanze per svolgere i trasporti sia ordinari che di emergenza nell’intera provincia, ma con le ambulanze interne e quelle in possesso degli ospedali periferici si riesce a coprire solo in minima parte il fabbisogno: si rende quindi necessario rivolgersi alle Organizzazioni di Volontariato che, in cambio di un rimborso spese, prestano ambulanze e soccorritori per lo svolgimento degli interventi. 
L’obiettivo del progetto è quello di minimizzare il costo per il mantenimento del servizio, garantendo comunque un pronto intervento in caso di necessità. 

\subsection{Problema Generale}
Ogni giorno la Centrale Operativa del 118 di Padova e l'Azienda Ospedaliera ricevono una moltitudine di chiamate, sia per richieste di supporto immediato (che vengono gestite dal SUEM), sia per trasporti programmati (che passano per le linee dell'Azienda Ospedaliera e che non rappresentano casi di emergenza, ma dedicati al trasporto di persone invalide presso l'ospedale per le visite programmate oppure per le dimissioni dal Pronto Soccorso).
\newline \newline
L'Azienda Ospedaliera, al suo interno, dispone di un determinato numero di ambulanze di tipo B (dedicate ai trasporti), mentre quelle di tipo A (dedicate al SUEM) sono fornite dagli ospedali periferici. L'insieme di questi mezzi non è però sufficiente a coprire le richieste pervenute dall'intera provincia, ma ne ricopre una minima parte. Per questo motivo, la Centrale Operativa si rivolge a tre Organizzazioni di Volontariato (OdV) che offrono il loro supporto per coprire i viaggi scoperti:
\begin{itemize}
    \item La Croce Rossa può fornire solo ambulanze di tipo A
    \item La Croce Bianca può fornire solo ambulanze di tipo B
    \item La Croce Verde può fornire sia ambulanze di tipo A che di tipo B
\end{itemize}
Il numero di ambulanze complessive in un determinato giorno viene stimato rispetto alla media di interventi e trasporti effettuati negli stessi giorni delle settimane precedenti, mantenendo comunque un numero arbitrario di ambulanze sia di tipo A che di tipo B in più.
\newline \newline
Si vuole quindi minimizzare il costo complessivo settimanale per il servizio, sapendo che:
\begin{itemize}
    \item Ogni giorno la Centrale Operativa ha necessità di un quantitativo diverso di ambulanze
    \item Qualora si attivasse una OdV per l'assistenza settimanale, è prevista una singola quota fissa di attivazione
    \item Ogni ambulanza richiamata dalle OdV prevede un rimborso spese giornaliero che varia in base all'organizzazione e al tipo di ambulanza attivata
    \item Le ambulanze dell'Azienda Ospedaliera e degli ospedali periferici, se attivate in un determinato giorno, prevedono dei piccoli costi di manutenzione
    \item Ogni OdV e ospedale dispone di un numero limitato di ambulanze
    \item Ogni giorno bisogna attivare un numero arbitrario di ambulanze in più rispetto al fabbisogno
    \item L'Ospedale deve attivare un minimo di fornitori ogni giorno
    \item Le OdV possono intervenire per un numero limitato di giorni
\end{itemize}
\vspace*{3 cm}
\textbf{Disclaimer: } \textit{questo progetto riguarda un tema reale con dati fittizi. Le Organizzazioni di Volontariato citate nell'introduzione e l'Ospedale di Padova sono realmente esistenti, ma i dati riguardanti disponibilità di mezzi, stima della necessità, metodologie di intervento e costi sostenuti sono casuali e adattati per rendere consistente lo studio in questione. Ogni riferimento è quindi da ritenersi puramente casuale in quanto frutto dell'immaginazione del redattore.}

\newpage
\section{Modello}
\subsection{Insiemi}
\begin{itemize}
    \item $Giorni$: giorni della settimana in cui è attivo il servizio
    \item $FornitoriA$: fornitori che dispongono di ambulanze di tipo A da attivare
    \item $FornitoriB$: fornitori che dispongono di ambulanze di tipo B da attivare
\end{itemize}
\subsection{Parametri}
\begin{itemize}
    \item $bisognoA_{g}$: fabbisogno per il giorno $g$ di ambulanze di tipo A (escluso surplus arbitrario)
    \item $bisognoB_{g}$: fabbisogno per il giorno $g$ di ambulanze di tipo B (escluso surplus arbitrario)
    \item $surplusA_{g}$: surplus di ambulanze di tipo A per il giorno $g$
    \item $surplusB_{g}$: surplus di ambulanze di tipo B per il giorno $g$
    \item $maxA_{fa}$: numero massimo di ambulanze di tipo A che il fornitore $fa$ può fornire
    \item $maxB_{fb}$: numero massimo di ambulanze di tipo B che il fornitore $fb$ può fornire
    \item $costoGiornalieroA_{fa}$: costo per l'attivazione giornaliera per una singola ambulanza di tipo A del fornitore $fa$
    \item $costoGiornalieroB_{fb}$: costo per l'attivazione giornaliera per una singola ambulanza di tipo B del fornitore $fb$
    \item $costoAttivazioneA_{fa}$: costo settimanale per l'attivazione di un fornitore $fa$ per la fornitura settimanale di ambulanze di tipo A
    \item $costoAttivazioneB_{fb}$: costo settimanale per l'attivazione di un fornitore $fb$ per la fornitura settimanale di ambulanze di tipo B
    \item $BigM$: utilizzato per vincolare l'utilizzo delle ambulanze rispetto all'attivazione del fornitore (costante sufficientemente grande), Questo, nei file \textbf{.dat} è stato definito utilizzando la più vicina potenza del 2 rispetto a $Max(bisognoA, bisognoB) + Max(surplusA, surplusB)$.
\end{itemize}
\subsection{Variabili Decisionali}
\begin{itemize}
    \item $ambulanzeA_{fa, ga}$ = numero di ambulanze del fornitore $fa$ attivate il giorno $ga$;
    \item $ambulanzeB_{fb, gb}$ =  numero di ambulanze del fornitore $fb$ attivate il giorno $gb$;
    \item $attivazioneSettimanaleA_{fa}$ = 
    \[
    \begin{cases}
        1 & \text{se viene attivato il fornitore di ambulanze di tipo A \textit{fa} per la settimana} \\
        0 & altrimenti
    \end{cases}\]
    \item $attivazioneSettimanaleB_{fb}$ = 
    \[
    \begin{cases}
        1 & \text{se viene attivato il fornitore di ambulanze di tipo B \textit{fb} per la settimana} \\
        0 & altrimenti
    \end{cases}\]
\end{itemize}

\subsection{Funzione Obiettivo}
La funzione obiettivo del problema può essere espressa in forma generale come segue:
\begin{align*}
	\textrm{min} \quad & \textbf{Costo\ Giornaliero\ Ambulanze\ Tipo\ A} + \textbf{Costo\ Giornaliero\ Ambulanze\ Tipo\ B} \\ 
    & + \textbf{Costo\ Attivazione\ Ambulanze\ Tipo\ A} + \textbf{Costo\ Attivazione\ Ambulanze\ Tipo\ B}
\end{align*}
Dove i termini sono così definiti:
\begin{flalign*}
    & \textbf{Costo\ Giornaliero\ Ambulanze\ Tipo\ A} = \sum_{f \in FornitoriA, g \in Giorni} ambulanzeA[f, g] \cdot costoGiornalieroA[f] && \\
    & \textbf{Costo\ Giornaliero\ Ambulanze\ Tipo\ B} = \sum_{f \in FornitoriB, g \in Giorni} ambulanzeB[f, g] \cdot costoGiornalieroB[f] && \\
    & \textbf{Costo\ Attivazione\ Ambulanze\ Tipo\ A} = \sum_{f \in FornitoriA} attivazioneSettimanaleA[f] \cdot costoAttivazioneA[f] && \\
    & \textbf{Costo\ Attivazione\ Ambulanze\ Tipo\ B} = \sum_{f \in FornitoriB} attivazioneSettimanaleB[f] \cdot costoAttivazioneB[f] &&
\end{flalign*}
\textbf{subject to}
\begin{itemize}
    \item Ogni giorno la Centrale Operativa ha necessità di un numero diverso di ambulanze:
\end{itemize}
\begin{flalign*}
    &  \sum_{f \in FornitoriA} ambulanzeA[f, g] \geq bisognoA[g] + surplusA[g] & \forall g \in Giorni & \\
    &  \sum_{f \in FornitoriB} ambulanzeB[f, g] \geq bisognoB[g] + surplusB[g] & \forall g \in Giorni &
\end{flalign*}
\begin{itemize}
    \item I fornitori dispongono di un numero massimo di ambulanze fornibili durante la settimana:
\end{itemize}
\begin{flalign*}
    &  ambulanze[f, g] \leq maxA[f] & \forall f \in FornitoriA, g \in Giorni & \\
    &  ambulanze[f, g] \leq maxB[f] & \forall f \in FornitoriB, g \in Giorni &
\end{flalign*}
\begin{itemize}
    \item Le ambulanze di un determinato fornitore si possono attivare se e solo se si è pagata la quota di attivazione settimanale:
\end{itemize}
\begin{flalign*}
    & ambulanzeA[f, g] \leq BigM \cdot attivazioneSettimanaleA[f] & \\
    & ambulanzeB[f, g] \leq BigM \cdot attivazioneSettimanaleB[f] &
\end{flalign*}
\textbf{Domini:}
\begin{itemize}
    \item $ambulanzeA_{f, g}, \ ambulanzeB_{f, g} \ \geq \ 0$
    \item $attivazioneSettimanaleA_{f}, \ attivazioneSettimanaleB_{f} \ \in \ {0, \ 1}$
\end{itemize}

\newpage
\section{Codice AMPL}
\subsection{File \textit{.mod}}
\newpage
\section{Scenari}
\subsection{Primo Scenario}
\subsubsection{Descrizione}
In questo primo scenario, i dati sono stati inseriti manualmente, con un numero di ambulanze necessarie tale da permettere di trovare una soluzione ottima.
\subsubsection{Codice}
\begin{lstlisting}
### INSIEMI ###
set Giorni := lunedi martedi mercoledi giovedi venerdi sabato domenica;
set FornitoriA := rossa verde interna;
set FornitoriB := bianca verde interna;

### PARAMETRI ###
param bisognoA := 
lunedi 20 martedi 22 mercoledi 10 
giovedi 20 venerdi 30 sabato 30
domenica 15
;

param bisognoB := 
lunedi 3 martedi 5 mercoledi 2
giovedi 3 venerdi 4 sabato 2
domenica 0
;

param surplusA :=
lunedi 3 martedi 3 mercoledi 3
giovedi 2 venerdi 5 sabato 6
domenica 4
;

param surplusB :=
lunedi 1 martedi 3 mercoledi 1
giovedi 1 venerdi 2 sabato 1
domenica 1
;

param maxA :=
rossa 15 verde 20 interna 3
;

param maxB :=
bianca 5 verde 10 interna 8
;

param costoGiornalieroA :=
rossa 15 verde 18 interna 8
;

param costoGiornalieroB :=
bianca 20 verde 12 interna 5
;

param costoAttivazioneA :=
rossa 130 verde 50 interna 0
;

param costoAttivazioneB :=
bianca 70 verde 55 interna 0
; 

param BigM := 64;
\end{lstlisting}
\subsubsection{Output}
L'esecuzione del file \textbf{.run} mostra che il costo minimo è di \texteuro 2929

\subsection{Secondo Scenario}
\subsubsection{Descrizione}
Il secondo set di dati è stato costruito manualmente con una richiesta di ambualanze non soddisfabile, impedendo quindi di trovare una soluzione.
\subsubsection{Codice}
\begin{lstlisting}
### INSIEMI ###
set Giorni := lunedi martedi mercoledi giovedi venerdi sabato domenica;
set FornitoriA := rossa verde interna;
set FornitoriB := bianca verde interna;

### PARAMETRI ###
param bisognoA := 
lunedi 150 martedi 22 mercoledi 10 
giovedi 20 venerdi 30 sabato 30
domenica 15
;

param bisognoB := 
lunedi 65 martedi 5 mercoledi 2
giovedi 3 venerdi 4 sabato 2
domenica 0
;

param surplusA :=
lunedi 3 martedi 3 mercoledi 3
giovedi 2 venerdi 5 sabato 6
domenica 4
;

param surplusB :=
lunedi 1 martedi 3 mercoledi 1
giovedi 1 venerdi 2 sabato 1
domenica 1
;

param maxA :=
rossa 15 verde 20 interna 3
;

param maxB :=
bianca 5 verde 10 interna 8
;

param costoGiornalieroA :=
rossa 15 verde 18 interna 8
;

param costoGiornalieroB :=
bianca 20 verde 12 interna 5
;

param costoAttivazioneA :=
rossa 130 verde 50 interna 0
;

param costoAttivazioneB :=
bianca 70 verde 55 interna 0
; 

param BigM := 256;
\end{lstlisting}
\subsubsection{Output}
Ovviamente, il costo minimo trovato eseguendo il file \textbf{.run} sarà pari a \texteuro 0.

\subsection{Terzo Scenario}
\subsubsection{Descrizione}
Nel terzo scenario sono stati definiti soltato gli insiemi, così da poter sfruttare i valori di default.
\subsubsection{Codice}
\begin{lstlisting}
### INSIEMI ###
set Giorni := lunedi martedi mercoledi giovedi venerdi sabato domenica;
set FornitoriA := rossa verde interna;
set FornitoriB := bianca verde interna;

### PARAMETRI ###
# Non definiti per test valori di default
\end{lstlisting}
\subsubsection{Output}
Eseguendo il file \textbf{.run} e utilizzando quindi i dati di default, il costo minimo risulta essere di \texteuro 350.
\end{document}