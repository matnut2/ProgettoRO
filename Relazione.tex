\documentclass[12pt, a4paper]{article}
\usepackage[utf8]{inputenc}
\usepackage{hyperref}   % PER I LINK INTERNI ED ESTERNI
\usepackage{graphicx}   % PER LE IMMAGINI
\usepackage{amsmath}    % PER LE FRAZIONI
\usepackage{listings}   % PER IL CODICE
\usepackage{xcolor}     % PER COLORARE IL CODICE

\graphicspath{{img/}}

\hypersetup{
    colorlinks=true,
    linkcolor=black,
    urlcolor=cyan,
    }

\title{Progetto di Ricerca Operativa}
\author{Soldà Matteo }
\date{A.A. 2022-2023}

\begin{document}

\begin{figure}
    \centering
    \includegraphics[width=0.50\textwidth]{Logo_Università_Padova}
\end{figure}

\maketitle

\newpage
\tableofcontents

\newpage
\section{Introduzione}
\subsection{Abstract}
\subsection{Problema Generale}
\newpage
\section{Modello}
\subsection{Insiemi}
\subsection{Parametri}
\subsection{Variabili Decisionali}
\subsection{Funzione Obiettivo}
\newpage
\section{Codice AMPL}
\subsection{File .mod}
\newpage
\section{Scenari}
\subsection{Primo Scenario}
\subsection{Secondo Scenario}
\subsection{Terzo Scenario}
\end{document}